%! Author =
%! Date = 19.11.2025

% Preamble
\documentclass[11pt]{article}

% Packages
\usepackage{amsmath}
\usepackage{graphicx}
\usepackage{booktabs}
\usepackage{amssymb}


% Document
\begin{document}

\section*{Tutorial 1 — Unsupervised Machine Learning (HS25)}

All code was written in PyCharm 2025.2.4 with the use of built-in AI assistant

\subsection{Part 1}
\item Interpolated spectra follow the original curves closely only with small deviations seen where interpolation smooths noise.
\item Along spectral type A→M, spectra change shape: hot stars have more flux at shorter wavelengths. Cooler stars peak at longer wavelengths, with molecular bands appearing.
\begin{figure}[htbp]
    \centering
    \includegraphics[width=1.1\textwidth]{./results/part1_vstars_original_vs_interpolated}
    \caption{Original vs interpolated spectra}
    \label{fig:part 1}
\end{figure}

\newpage
\subsection{Part 2}
As PCA is based on covariance, standardizing ensures no wavelength dominates purely because of larger variance.
We normalize each spectrum to [0,1] to remove absolute brightness, then standardize per wavelength to equalize variance across wavelengths so PCA doesn’t overweight certain regions.
This should be consistent with standard PCA practice.

\begin{figure}[htbp]
    \centering
    \includegraphics[width=0.9\textwidth]{./results/part2_pca_scree}
    \caption{Variance vs number of PCs, with a horizontal line at 95\%}
    \label{fig:part 2 scree}
\end{figure}

\item As it can be seen below spectral types form ordered bands along PC1 (hot to cool), but overlap, which gives us a continuous sequence and not perfectly separated clusters. Luminosity classes overlap much more.
\item PC1+PC2 explain $0.918$ 2D captures main trend, but not with all details ($<95\%$).
\item PC3 adds significant variance $0.966$ ($>95\%$) and helps separate luminosity classes, but 2D is still easier to visualize.

\begin{figure}[htbp]
    \centering
    \includegraphics[width=0.8\textwidth]{./results/part2_pca2_scatter_by_sptype}
    \caption{PCA(n\_components=2) scatter colored by spectral type}
    \label{fig:part 2 sptype}
\end{figure}

\begin{figure}[htbp]
    \centering
    \includegraphics[width=0.8\textwidth]{./results/part2_pca2_scatter_by_luminosity}
    \caption{PCA(n\_components=2) scatter colored by luminosity class}
    \label{fig:part 2 luminosity}
\end{figure}

\newpage
\subsection{Part 3}

\item As it can be seen on the plot below Inertia monotonically decreases with k.
\item Silhouette and Calinski–Harabasz show several local maxima. Davies–Bouldin lowers with incresing k.
\item So we can say that the best k is not unique as several values have similar internal scores.

\item NOTE: This part is unfinished because something is wrong in calculations and with plotting. But below there is Part 4.


\begin{figure}[htbp]
    \centering
    \includegraphics[width=0.8\textwidth]{./results/part3_kmeans_model_selection}
    \caption{}
    \label{fig:part 3 kmeans}
\end{figure}

\begin{figure}[htbp]
    \centering
    \includegraphics[width=0.5\textwidth]{./results/part3_kmeans_scatter}
    \caption{}
    \label{fig:part 3 kmeans scatter}
\end{figure}

\begin{figure}[htbp]
    \centering
    \includegraphics[width=0.5\textwidth]{./results/part3_dbscan_scatter}
    \caption{}
    \label{fig:part 3 dbscan scatter}
\end{figure}

\begin{figure}[htbp]
    \centering
    \includegraphics[width=0.5\textwidth]{./results/part3_dbscan_kdist_curve}
    \caption{}
    \label{fig:part 3 kdist curve}
\end{figure}

%\begin{figure}[htbp]
%    \centering
%    \includegraphics[width=0.5\textwidth]{./results/part3_dbscan_silhouette_vs_nclusters}
%    \caption{}
%    \label{fig:part 3 silhouette vs nclusters}
%\end{figure}

\newpage
\subsection{Part 4}
\item With different methods we get different outliners.

\begin{table}[htbp!]
    \centering
    \tiny
    \begin{tabular}{rccc}
        \toprule
        Object ID & Robust z-score (top 1\%) & Isolation Forest (top 1\%) & LOF (top 1\%) \\
        \midrule
        82   &              & \checkmark &             \\
        333  &              &            & \checkmark  \\
        346  & \checkmark   &            &             \\
        509  &              & \checkmark &             \\
        542  &              & \checkmark &             \\
        747  &              &            & \checkmark  \\
        756  &              & \checkmark &             \\
        766  & \checkmark   &            &             \\
        838  &              & \checkmark & \checkmark  \\
        871  &              & \checkmark &             \\
        892  &              & \checkmark & \checkmark  \\
        909  &              & \checkmark & \checkmark  \\
        988  &              & \checkmark &             \\
        1033 &              &            & \checkmark  \\
        1034 &              &            & \checkmark  \\
        1105 &              &            & \checkmark  \\
        1144 &              & \checkmark & \checkmark  \\
        1181 &              & \checkmark & \checkmark  \\
        1199 & \checkmark   &            &             \\
        1266 &              & \checkmark &             \\
        1269 &              &            & \checkmark  \\
        1339 &              & \checkmark & \checkmark  \\
        1359 & \checkmark   &            &             \\
        1367 & \checkmark   &            &             \\
        1383 &              & \checkmark & \checkmark  \\
        1388 &              & \checkmark &             \\
        1451 &              & \checkmark & \checkmark  \\
        1452 &              &            & \checkmark  \\
        1543 & \checkmark   & \checkmark & \checkmark  \\
        1551 &              &            & \checkmark  \\
        1553 &              &            & \checkmark  \\
        1554 &              & \checkmark & \checkmark  \\
        1561 & \checkmark   &            & \checkmark  \\
        1670 &              &            & \checkmark  \\
        1812 &              &            & \checkmark  \\
        2123 &              & \checkmark &             \\
        2125 &              & \checkmark &             \\
        2130 &              & \checkmark &             \\
        2140 & \checkmark   &            &             \\
        2141 &              & \checkmark &             \\
        2147 & \checkmark   &            &             \\
        2155 & \checkmark   &            &             \\
        2159 & \checkmark   &            &             \\
        2174 & \checkmark   &            &             \\
        2176 & \checkmark   &            &             \\
        2177 & \checkmark   &            &             \\
        2178 & \checkmark   &            &             \\
        2180 & \checkmark   &            &             \\
        2189 & \checkmark   &            &             \\
        2192 & \checkmark   &            &             \\
        2198 & \checkmark   &            &             \\
        2203 & \checkmark   &            &             \\
        2207 & \checkmark   & \checkmark &             \\
        2208 & \checkmark   &            &             \\
        2209 & \checkmark   &            & \checkmark  \\
        \bottomrule
    \end{tabular}
    \caption{Objects detected by each anomaly detection method (checkmarks indicate detection)}
    \label{tab:method_overlap}
\end{table}
\item Robust z-score (R): 23 objects. Isolation Forest (I): 23 objects. LOF (L): 23 objects.
\item Total unique objects $(R$ U $I$ U $L)$: 55.
\newpage

\begin{figure}[htbp]
    \centering
    \includegraphics[width=0.9\textwidth]{./results/part4_outliers_robust_zscore}
    \caption{Ouliners with z-score method}
    \label{fig:part 4 zscore}
\end{figure}

\begin{figure}[htbp]
    \centering
    \includegraphics[width=0.9\textwidth]{./results/part4_outliers_isolation_forest}
    \caption{Ouliners with isolation forest method}
    \label{fig:part 4 isolation forest}
\end{figure}

\begin{figure}[htbp]
    \centering
    \includegraphics[width=0.9\textwidth]{./results/part4_outliers_lof}
    \caption{Ouliners with local outlier factor method}
    \label{fig:part 4 lof}
\end{figure}

\end{document}